\PassOptionsToPackage{usenames}{color}
\documentclass[12pt,a4paper]{article}
\usepackage[margin=1in]{geometry}

\usepackage{relsize} % relative font sizes (e.g. \smaller). must precede ACL style
%\usepackage{style/acl2012}
\usepackage[colorlinks=true,linkcolor=black,citecolor=black,filecolor=black,urlcolor=black]{hyperref}
\usepackage{natbib}

%\usepackage{times}
%\usepackage{latexsym}

\usepackage{microtype}

\usepackage[boxed]{algorithm2e}
\renewcommand\AlCapFnt{\small}
\usepackage[small,bf,skip=5pt]{caption}
\usepackage{sidecap} % side captions
\usepackage{rotating}	% sideways

% Italicize subparagraph headings
\usepackage{titlesec}
\titleformat*{\section}{\large\larger[.5]\bfseries}
\titleformat*{\subsection}{\large\bfseries}
\titleformat*{\subsubsection}{\large\bfseries}
\titleformat*{\paragraph}{\large\bfseries}
\titleformat*{\subparagraph}{\itshape}
\titlespacing{\subparagraph}{%
  1em}{%              left margin
  0pt}{% space before (vertical)
  1em}%               space after (horizontal)

% Numbered Examples and lists
\usepackage{lingmacros}

\usepackage{enumitem} % customizable lists
\setitemize{noitemsep,topsep=0em}
\setenumerate{noitemsep,leftmargin=0em,itemindent=13pt,topsep=0em}


\usepackage{textcomp}
% \usepackage{arabtex} % must go after xparse, if xparse is used!
%\usepackage{utf8}
% \setcode{utf8} % use UTF-8 Arabic
% \newcommand{\Ar}[1]{\RL{\novocalize #1}} % Arabic text

\usepackage{listings}

\lstset{
  basicstyle=\itshape,
  xleftmargin=3em,
  aboveskip=0pt,
  belowskip=-3pt, %-.5\baselineskip, % correct for extra paragraph break inserted after listing
  literate={->}{$\rightarrow$}{2}
           {α}{$\alpha$}{1}
           {δ}{$\delta$}{1}
           {(}{$($}{1}
           {)}{$)$}{1}
           {[}{$[$}{1}
           {]}{$]$}{1}
           {|}{$|$}{1}
           {+}{\ensuremath{^+}}{1}
           {*}{\ensuremath{^*}}{1}
}

\usepackage{amssymb}	%amsfonts,eucal,amsbsy,amsthm,amsopn
\usepackage{amsmath}

\usepackage{mathptmx}	% txfonts
\usepackage[scaled=.8]{beramono}
\usepackage[T1]{fontenc}
\usepackage[utf8x]{inputenc}

\usepackage{MnSymbol}	% must be after mathptmx

\usepackage{latexsym}





% Tables
\usepackage{array}
\usepackage{multirow}
\usepackage{booktabs} % pretty tables
\usepackage{wrapfig}
\usepackage{multicol}
\usepackage{footnote}

\usepackage{url}
\usepackage[usenames]{color}
\usepackage{xcolor}

% colored frame box
\newcommand{\cfbox}[2]{%
    \colorlet{currentcolor}{.}%
    {\color{#1}%
    \fbox{\color{currentcolor}#2}}%
}

\usepackage[normalem]{ulem} % \uline
\usepackage{colortbl}
\usepackage{graphicx}
\usepackage{subcaption}
%\usepackage{tikz-dependency}
\usepackage{tikz}
%\usepackage{tree-dvips}
\usetikzlibrary{arrows,positioning,calc} 

\DeclareMathOperator*{\argmax}{arg\,max}
\DeclareMathOperator*{\argmin}{arg\,min}
%\setlength\titlebox{6.5cm}    % Expanding the titlebox



% Author comments
\usepackage{color}
\newcommand\bmmax{0} % magic to avoid 'too many math alphabets' error
\usepackage{bm}
\definecolor{orange}{rgb}{1,0.5,0}
\definecolor{mdgreen}{rgb}{0,0.6,0}
\definecolor{mdblue}{rgb}{0,0,0.7}
\definecolor{dkblue}{rgb}{0,0,0.5}
\definecolor{dkgray}{rgb}{0.3,0.3,0.3}
\definecolor{slate}{rgb}{0.25,0.25,0.4}
\definecolor{gray}{rgb}{0.5,0.5,0.5}
\definecolor{ltgray}{rgb}{0.7,0.7,0.7}
\definecolor{purple}{rgb}{0.7,0,1.0}
\definecolor{lavender}{rgb}{0.65,0.55,1.0}

% Settings for algorithm listings
% \lstset{
%   language=Python,
%   upquote=true,
%   showstringspaces=false,
%   formfeed=\newpage,
%   tabsize=1,
%   commentstyle=\itshape\color{lavender},
%   basicstyle=\small\smaller\ttfamily,
%   morekeywords={lambda},
%   emph={upward,downward,tc},
%   emphstyle=\underbar,
%   aboveskip=0cm,
%   belowskip=-.5cm
% }
%\renewcommand{\lstlistingname}{Algorithm}


\newcommand{\ensuretext}[1]{#1}
\newcommand{\cjdmarker}{\ensuretext{\textcolor{green}{\ensuremath{^{\textsc{CJ}}_{\textsc{D}}}}}}
\newcommand{\nssmarker}{\ensuretext{\textcolor{magenta}{\ensuremath{^{\textsc{NS}}_{\textsc{S}}}}}}
\newcommand{\nasmarker}{\ensuretext{\textcolor{red}{\ensuremath{^{\textsc{NA}}_{\textsc{S}}}}}}
\newcommand{\bomarker}{\ensuretext{\textcolor{blue}{\ensuremath{^{\textsc{B}}_{\textsc{O}}}}}}
\newcommand{\jbmarker}{\ensuretext{\textcolor{orange}{\ensuremath{^{\textsc{J}}_{\textsc{B}}}}}}
\newcommand{\dbmarker}{\ensuretext{\textcolor{purple}{\ensuremath{^{\textsc{D}}_{\textsc{B}}}}}}
\newcommand{\arkcomment}[3]{\ensuretext{\textcolor{#3}{[#1 #2]}}}
%\newcommand{\arkcomment}[3]{}
\newcommand{\cjd}[1]{\arkcomment{\cjdmarker}{#1}{green}}
\newcommand{\nss}[1]{\arkcomment{\nssmarker}{#1}{magenta}}
\newcommand{\nas}[1]{\arkcomment{\nasmarker}{#1}{red}}
\newcommand{\bo}[1]{\arkcomment{\bomarker}{#1}{blue}}
\newcommand{\jb}[1]{\arkcomment{\jbmarker}{#1}{orange}}
\newcommand{\db}[1]{\arkcomment{\dbmarker}{#1}{purple}}
\newcommand{\wts}{\mathbf{w}}
\newcommand{\g}{\mathbf{g}}
\newcommand{\f}{\mathbf{f}}
\newcommand{\x}{\mathbf{x}}
\newcommand{\y}{\mathbf{y}}
\newcommand{\overbar}[1]{\mkern 1.5mu\overline{\mkern-1.5mu#1\mkern-1.5mu}\mkern 1.5mu} % \bar is too narrow in math
\newcommand{\cost}{c}

\newcommand{\Sref}[1]{\S\ref{#1}}
\newcommand{\fref}[1]{figure~\ref{#1}}
\newcommand{\ffref}[2]{figures~\ref{#1} and~\ref{#2}}
\newcommand{\Fref}[1]{Figure~\ref{#1}}
\newcommand{\FFref}[2]{Figures~\ref{#1} and~\ref{#2}}
\newcommand{\tref}[1]{table~\ref{#1}}
\newcommand{\ttref}[2]{tables~\ref{#1} and~\ref{#2}}
\newcommand{\Tref}[1]{Table~\ref{#1}}
\newcommand{\aref}[1]{algorithm~\ref{#1}}
\newcommand{\Aref}[1]{Algorithm~\ref{#1}}
\newcommand{\fnref}[1]{footnote~\ref{#1}}

\newcommand{\citeposs}[1]{\citeauthor{#1}'s (\citeyear{#1})}

% Space savers
% From http://www.eng.cam.ac.uk/help/tpl/textprocessing/squeeze.html
\addtolength{\dbltextfloatsep}{-.5cm} % space between last top float or first bottom float and the text.
\addtolength{\intextsep}{-.5cm} % space left on top and bottom of an in-text float.
\addtolength{\abovedisplayskip}{-.5cm} % space before maths
\addtolength{\belowdisplayskip}{-.5cm} % space after maths
%\addtolength{\topsep}{-.5cm} %space between first item and preceding paragraph
\setlength{\belowcaptionskip}{-.25cm}


% customize \paragraph spacing
\makeatletter
\renewcommand{\paragraph}{%
  \@startsection{paragraph}{4}%
  {\z@}{.2ex \@plus 1ex \@minus .2ex}{-1em}%
  {\normalfont\normalsize\bfseries}%
}
\makeatother



% Special macros
\newcommand{\tg}[1]{\texttt{#1}}	% supersense tag name
\newcommand{\gfl}[1]{%\renewcommand\texttildelow{{\lower.74ex\hbox{\texttt{\char`\~}}}} % http://latex.knobs-dials.com/
\mbox{\textsmaller{\texttt{#1}}}}	% supersense tag symbol
\newcommand{\lex}[1]{\textsmaller{\textsf{\textcolor{slate}{\textbf{#1}}}}}	% example lexical item 
\newcommand{\tagdef}[1]{#1\hfill} % tag definition
\newcommand{\tagt}[2]{\ensuremath{\underset{\textrm{\textlarger{\tg{#2}}}\strut}{\w{#1}\rule[-.3\baselineskip]{0pt}{0pt}}}} % tag text (a word or phrase) with an SST. (second arg is the tag)
\newcommand{\glosst}[2]{\ensuremath{\underset{\textrm{#2}}{\textrm{#1}}}} % gloss text (a word or phrase) (second arg is the gloss)
\newcommand{\AnnA}[0]{\mbox{\textbf{Ann-A}}} % annotator A
\newcommand{\AnnB}[0]{\mbox{\textbf{Ann-B}}} % annotator B
\newcommand{\sys}[1]{\mbox{\textbf{#1}}}   % name of a system (one of our experimental conditions)
\newcommand{\dataset}[1]{\mbox{\textsc{#1}}}	% one of the datasets in our experiments
\newcommand{\datasplit}[1]{\mbox{\textbf{#1}}}	% portion one of the datasets in our experiments

\newcommand{\w}[1]{\textit{#1}}	% word
\newcommand{\gap}[0]{\ \ } % space around gap contents
\newcommand{\tat}[0]{\textasciitilde}

\newcommand{\shortlong}[2]{#1} % short vs. long version of the paper
\newcommand{\confversion}[1]{#1}
\newcommand{\srsversion}[1]{}
\newcommand{\finalversion}[1]{#1}
%\newcommand{\finalversion}[1]{}
\newcommand{\shortversion}[1]{#1}
\newcommand{\considercutting}[1]{#1}
\newcommand{\longversion}[1]{} % ...if only there were more space...
\newcommand{\subversion}[1]{#1} % for the submission version only
\newcommand{\draftnotice}[1]{} % for the draft version only
%\newcommand{\subversion}[1]{}

\hyphenation{WordNet}
\hyphenation{WordNets}
\hyphenation{VerbNet}
\hyphenation{FrameNet}
\hyphenation{SemCor}
\hyphenation{PennConverter}
\hyphenation{an-aly-sis}
\hyphenation{an-aly-ses}
\hyphenation{news-text}
\hyphenation{base-line}
\hyphenation{de-ve-lop-ed}
\hyphenation{comb-over}

\title{The Logic of AMR: Practical, Unified, \\ Graph-Based Sentence Semantics for NLP}

\author{\textbf{Nathan Schneider}\\
	University of Edinburgh\\
	{\tt nschneid@inf.ed.ac.uk}
\and
	\textbf{Jeffrey Flanigan}\\
	Carnegie Mellon University\\
	{\tt jflanigan@cs.cmu.edu}
	    }

% \author{Author 1\\
% 	    XYZ Company\\
% 	    111 Anywhere Street\\
% 	    Mytown, NY 10000, USA\\
% 	    {\tt author1@xyz.org}
% 	  \And
% 	Author 2\\
%   	ABC University\\
%   	900 Main Street\\
%   	Ourcity, PQ, Canada A1A 1T2\\
%   {\tt author2@abc.ca}}

\date{}

\begin{document}
\maketitle

\abstract{\noindent
This tutorial will provide a detailed introduction to the Abstract Meaning Representation (AMR) formalism 
and its use for sentence semantics in NLP. Our goals are twofold. 
First, we will describe the nature and design principles behind the representation, 
and demonstrate that it can be practical for annotation. Participants will be coached in the basics of annotation 
so they will be prepared to work with existing data with an understanding of the benefits and limitations 
of the process by which it was created. 
Second, we will survey the state of the art for computation with AMRs. 
This will focus on the task of parsing English text into AMR graphs, which 
requires algorithms for alignment, for structured prediction, and for statistical learning. 
Advances toward AMR-based machine translation and other multilingual studies\nss{vague} will be reviewed as well.}

\section{Introduction}

The Abstract Meaning Representation formalism \citep[AMR;]{banarescu-13} 
is rapidly emerging as an important practical form of structured sentence semantics. 
Conceived just a few years ago, AMR has enjoyed momentum thanks to 
the ongoing development of large-scale annotated corpora 
(\nss{\#} words annotated; \nss{\#} expected by the end of 2015).
It has stimulated research in automata-theoretic formalisms \citep{jones-12,chiang-13}
as well as structured prediction algorithms \citep{flanigan-14}.
Following on the heels of the Fred Jelinek Memorial Workshop in Prague (summer, 2014), 
which explored the cross-lingual linguistic and computational implications of AMR, 
this tutorial unmasks the design philosophy, data creation process, and existing algorithms for 
AMR semantics.

The proposed tutorial structure is in two main parts. 
Roughly speaking, the first half will be devoted to linguistic and annotation matters, 
while the second half will focus on algorithms and applications.
By the end of the first half, participants will have a firm grasp of the AMR representation, 
which will help motivate the technical challenges of the second half.

Below, we give an extremely brief description of how AMR works and why it has potential 
as a convergence point\nss{?} for NLP research. 
That will be followed by a high-level overview of the tutorial goals and structure, 
as well as a tentative schedule.

\section{Why AMR?}

\section{Agenda}

\begin{enumerate}
\item The AMR formalism for English \begin{itemize}
	\item How meaning is structured (20\%)
	\item What is not represented, and how AMR differs from alternatives (10\%)
	\item Annotation tools and data resources (5\%)
	\item Guided annotation practice with the AMR Editor (10\%)
	\end{itemize}
\item Algorithms and applications \begin{itemize}
	\item AMR alignment, parsing, and evaluation (30\%)
	\item Multilinguality, machine translation, and other applications (15\%)
	\item Open challenges, criticisms, and limitations (5\%)
	\end{itemize}
\end{enumerate}

\section{Planning}

The authors will hold dry runs of the tutorial at their respective institutions, 
which will be especially important for debugging the hands-on activities. 

\nss{venue; attendance; equipment required}

\section{Instructors}

\textbf{Nathan Schneider} is an annotation schemer and computational modeler for natural language. 
He has been involved in the design of the AMR formalism since 2012, 
when he interned with Kevin Knight at ISI. 
His dissertation introduced a coarse-grained representation for lexical semantics that facilitates rapid annotation 
and is practical for broad-coverage statistical NLP \citep{schneider-thesis}. 
He has also worked on semantic parsing for the FrameNet representation \citep{das-14} 
and other forms of annotation and processing for social media text \citep{gimpel-11,owoputi-13,schneider-13,kong-14,mohit-12}.
For most of these projects, he led the design of the annotation scheme, guidelines, and workflows, 
and the training and supervision of annotators.
He will lead the first half of the tutorial.

\textbf{Jeffrey Flanigan} \nss{\ldots MT, built the FIRST EVAR realistic AMR parser, attended summer workshop}

\bibliographystyle{aclnat}
% you bib file should really go here
\setlength{\bibsep}{1pt}
{\fontsize{10}{12.25}\selectfont
\bibliography{amr}}

\end{document}
